\chapter{Einleitung}
\label{cha:Einleitung}

\section{Motivation}
\label{sec:Einl:Motivation}


\section{Ziel der Arbeit}
\label{sec:Einl:Ziel_der_Arbeit}



\section{Überblick über den Aufbau der Arbeit}
\label{sec:Einleitung:Aufbau_der_Arbeit}



\section{Typographische Konventionen}
\label{sec:Typographische_Konventionen}

Zum besseren Verständnis dieser Arbeit werden einige typographische Konventionen festgelegt.

Fachbegriffe werden \Fachbegriff{kursiv} formatiert.

Klassennamen und einzeilige Codefragmente werden in \Code{Proportionalschrift}, längerer Quelltext in Form von Codeblöcken, die als Listings bezeichnet werden, dargestellt.

Zitate und Metaphern werden in ">doppelte Anführungszeichen"< gestellt.

Liegt eine besondere Betonung auf einem Wort, so wird dieses \textbf{fettgedruckt} dargestellt.
Sonstige Hervorhebungen werden ebenfalls \textbf{fettgedruckt}.

Abkürzungen werden bei erster Nennung kurz erläutert und können zudem im Abkürzungsverzeichnis auf Seite \pageref{sec:Glossar} nachgeschlagen werden.